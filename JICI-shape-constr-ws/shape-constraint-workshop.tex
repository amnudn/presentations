% Created 2023-09-08 Fri 09:19
% Intended LaTeX compiler: pdflatex
\documentclass[smaller]{beamer}\usepackage{listings}
\usepackage{color}
\usepackage{amsmath}
\usepackage{array}
\usepackage[T1]{fontenc}
\usepackage{natbib}
\lstset{
keywordstyle=\color{blue},
commentstyle=\color{red},stringstyle=\color[rgb]{0,.5,0},
literate={~}{$\sim$}{1},
basicstyle=\ttfamily\small,
columns=fullflexible,
breaklines=true,
breakatwhitespace=false,
numbers=left,
numberstyle=\ttfamily\tiny\color{gray},
stepnumber=1,
numbersep=10pt,
backgroundcolor=\color{white},
tabsize=4,
keepspaces=true,
showspaces=false,
showstringspaces=false,
xleftmargin=.23in,
frame=single,
basewidth={0.5em,0.4em},
}
\usepackage{natbib, dsfont, pgfpages, tikz,amssymb, amsmath,xcolor}
\bibliographystyle{abbrvnat}
\input{./latex-settings/standard-commands.tex}
\setbeamertemplate{footline}[frame number]
\beamertemplatenavigationsymbolsempty
\usepackage{appendixnumberbeamer}
\setbeamercolor{gray}{bg=white!90!black}
\setbeamertemplate{itemize items}{$\circ$}

\renewcommand*\familydefault{\sfdefault}
\itemsep2pt
\usepackage[utf8]{inputenc}
\usepackage[T1]{fontenc}
\usepackage{graphicx}
\usepackage{longtable}
\usepackage{wrapfig}
\usepackage{rotating}
\usepackage[normalem]{ulem}
\usepackage{amsmath}
\usepackage{amssymb}
\usepackage{capt-of}
\usepackage{hyperref}
\usetheme{default}
\author{Anders Munch and Thomas Gerds}
\date{September 11, 2023}
\title{Targeted learning under shape constraints}
\begin{document}

\maketitle
\begin{frame}[label={sec:org49c2258}]{Motivation}
The road map of causal learning tells us to incorporate all the knowledge that we have
  into the statistical model for the distribution of the data
\vfill

In many real applications, subject matter knowledge is available
  regarding the shape of the underlying conditional density and
  regression functions
\vfill

Examples of biologically motivated shapes are
\begin{itemize}
\item risk of disease is not decreasing with age (given other covariates)
\item The risk of disease should be a monotone function of age (given other covariates)
\item The number of comorbidities increases the risk of disease
\item the effect of a biomarker on the risk of disease is an unimodal function (given other covariates)
\end{itemize}
\end{frame}

\begin{frame}[label={sec:org4f80621}]{}
\begin{block}{Working hypotheses}
\begin{itemize}
\item Shape constraints can be incorporated into machine learning for nuisance parameters
\item Biologically motivated shape constraints 
may lead to improved estimators
\end{itemize}

\vspace{4em}
\end{block}

\begin{block}{Goal for the workshop}
\begin{itemize}
\item Discuss some initial hypotheses and ideas
\item Help us move in the right research direction
\end{itemize}
\end{block}
\end{frame}

\begin{frame}[label={sec:orgf6efc36}]{Multivariate shape constraints}
\begin{figure}[htbp]
\centering
\includegraphics[width=0.7\textwidth]{./a.pdf}
\label{fig:1}
\end{figure}
\end{frame}

\begin{frame}[label={sec:org95b485a}]{Machine learning (from the shelf)}
\begin{figure}[htbp]
\centering
\includegraphics[width=0.7\textwidth]{./b.pdf}
\label{fig:1}
\end{figure}
\end{frame}
\begin{frame}[label={sec:org5a17eb4}]{Shape constraints}
\begin{block}{Examples}
\begin{itemize}
\item Monotonicity
\item Unimodality
\item Convexity
\item Log-concave densities
\end{itemize}
\end{block}

\begin{block}{Constraints imposed on target or nuisance parameter}
Shape constraint on a function-valued target parameter has been
considered
\citep[e.g.,][]{groeneboom2014nonparametric,westling2020unified,wu2022nonparametric}. We
will mostly discuss imposing shape-constraints on nuisance parameters.
\end{block}
\end{frame}

\begin{frame}[label={sec:org4e28a00}]{Information bounds}
\begin{alertblock}{Claim 1}
Most shape constraints will not restrict the tangent space, and hence imposing
shape constraints does not change the information bound for a statistical
estimation problem.

\begin{itemize}
\item Which shape constraints (if any) is this true for?
\item Can we still expect to improve a TMLE by imposing shape constraints on the
nuisance parameters?
\end{itemize}

\pause
\end{alertblock}

\begin{alertblock}{Claim 2}
Constructing a TMLE under a shape constrained model will typically result in a
sub-model that is not contained in the shape constrained model.

\begin{itemize}
\item Is this a problem?
\end{itemize}
\end{alertblock}
\end{frame}

\begin{frame}[label={sec:org5fa90df}]{(Un)necessary restrictions on nuisance parameters?}
\begin{block}{Undersmoothing}
It has been argued that undersmoothing estimators of nuisance parameters can
provide better estimators of a low-dimensional target parameter
\citep[e.g.,][]{goldstein1996efficient,hjort2001note,van2022efficient}. Could
shape constrained estimators provide unnecessary smoothing of nuisance parameter
estimators, which might in fact be damaging? \pause
\end{block}

\begin{block}{Biologically reasonable nuisance parameter estimators?}
Should we pay attention to whether nuisance parameters are estimated by
biologically meaningful estimators?

\hfill

Should we accept a biologically unreasonable estimator of a nuisance parameter
as long as it provides a good estimator of the target parameter?
\end{block}
\end{frame}


\begin{frame}[label={sec:orgfb1215a}]{Estimating a cumulative distribution function}
\small

\begin{description}
\item[{\color{red}ECDF}] Empirical distribution function
\item[{\color{orange}kernCDF}] Estimator based on smoothed kernel density estimator
\item[{\color{blue}logConCDF}] Estimator based on log-concave density estimator
\citep{dumbgen2009maximum,Rufibach_Duembgen_2023}
\end{description}

\includegraphics[width=1\textwidth]{./cdf-estimators.pdf}
\end{frame}


\begin{frame}[label={sec:org7d5017a}]{Challenges for future research}
\begin{itemize}
\item Should we distinguish between learning Q vs G parts of a causal
model/information loss model?

\item How do we translate ``marginal'' smoothness constraints into 
constraints on a multivariate function?

\item In longitudinal settings: need to discuss shape-constraints on the
history (filtration)
\end{itemize}
\end{frame}


\begin{frame}[label={sec:orgf5e37e8}]{References}
\footnotesize \bibliography{./latex-settings/default-bib.bib}
\end{frame}
\end{document}