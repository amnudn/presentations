% Created 2022-03-29 Tue 11:04
% Intended LaTeX compiler: pdflatex
\documentclass[11pt]{article}
\usepackage[utf8]{inputenc}
\usepackage[T1]{fontenc}
\usepackage{graphicx}
\usepackage{grffile}
\usepackage{longtable}
\usepackage{wrapfig}
\usepackage{rotating}
\usepackage[normalem]{ulem}
\usepackage{amsmath}
\usepackage{textcomp}
\usepackage{amssymb}
\usepackage{capt-of}
\usepackage{hyperref}
\usepackage{listings}
\usepackage{color}
\usepackage{amsmath}
\usepackage{array}
\usepackage[T1]{fontenc}
\usepackage{natbib}
\author{Anders Munch}
\date{\today}
\title{Model and tuning parameter selection using cross-validation on censored data}
\begin{document}

\maketitle

\section{Abstract}
\label{sec:orge250621}
We consider the problem of selecting a regression model from a collection of candidate models using
cross-validation on an external data set with censored observations in continuous time. In this
situation it is common to use the component of the negative log-likelihood corresponding to the
outcome as the loss function. This ignores the contribution to the likelihood made by the censoring
distribution, and we argue that this approach is problematic in at least two ways: Firstly, we show
that the least false parameter according to this loss function is in general not well-defined as it
depends on the censoring distribution. Secondly, for many common survival models (e.g., the Cox
model) the likelihood will a.s. be zero for any hold-out sample, meaning that this loss function
cannot be used to compare general survival models and hence does not provide a general approach for
model selection in the survival setting.

We show how modeling of the censoring distribution can alleviate these problems. On the other hand,
this introduces the problem of estimating the censoring distribution, which without further
assumptions is as equally complicated as estimating the outcome model. We make some comments on this
and consider the possibility of using other loss functions. Finally, we argue that it is worth
considering whether the final goal of the analysis is the selection of a well-performing risk
predicting model, or if the model selected is part of the a nuisance parameter estimation step. In
the latter case, different strategies for model selection could be considered.
\end{document}