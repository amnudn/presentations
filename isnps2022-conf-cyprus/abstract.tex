% Created 2022-06-11 Sat 15:01
% Intended LaTeX compiler: pdflatex
\documentclass[11pt]{article}
\usepackage[utf8]{inputenc}
\usepackage[T1]{fontenc}
\usepackage{graphicx}
\usepackage{longtable}
\usepackage{wrapfig}
\usepackage{rotating}
\usepackage[normalem]{ulem}
\usepackage{amsmath}
\usepackage{amssymb}
\usepackage{capt-of}
\usepackage{hyperref}
\usepackage{listings}
\usepackage{color}
\usepackage{amsmath}
\usepackage{array}
\usepackage[T1]{fontenc}
\usepackage{natbib}
\author{Anders Munch}
\date{\today}
\title{Challenges with model and tuning parameter selection when using cross-validation with censored data}
\begin{document}

\maketitle

\section{Abstract}
\label{sec:org304af21}
Most machine learning algorithms depend on one or more tuning parameters to control the trade-off
between bias and variance. To select the optimal value for a tuning parameter, a popular approach is
to use cross-validation to determine which value minimizes a given loss function. We consider the
problem of selecting a regression model from a collection of candidate models using cross-validation
on an external data set where the observations might be right-censored in continuous time. In this
situation it is common to use the component of the negative log-likelihood corresponding to the
outcome as the loss function. This ignores the contribution to the likelihood made by the censoring
distribution, and we argue that this approach is problematic in at least two ways: Firstly, we show
that the least false parameter according to this loss function is in general not well-defined as it
depends on the censoring distribution. Secondly, for many commonly used survival models the
likelihood will a.s. be zero for any hold-out sample. This means that the negative log-likelihood
loss cannot be used to compare general survival models and hence does not provide a general approach
for model selection in the survival setting. We discuss how these problems can be alleviated by
modeling the censoring distribution, which, on the other hand, comes at the cost of introducing a
new nuisance parameter to be estimated.

\section{Parts of old version}
\label{sec:orgdda95c1}

We show how modeling of the censoring distribution can alleviate these problems. On the other hand,
this introduces the problem of estimating the censoring distribution, which without further
assumptions is as equally complicated as estimating the outcome model. We make some comments on this
and consider the possibility of using other loss functions. Finally, we argue that it is worth
considering whether the final goal of the analysis is the selection of a well-performing risk
predicting model, or if the model selected is part of the a nuisance parameter estimation step. In
the latter case, different strategies for model selection could be considered.
\end{document}