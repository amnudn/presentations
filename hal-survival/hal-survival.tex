% Created 2023-02-14 Tue 22:52
% Intended LaTeX compiler: pdflatex
\documentclass[smaller]{beamer}\usepackage{listings}
\usepackage{color}
\usepackage{amsmath}
\usepackage{array}
\usepackage[T1]{fontenc}
\usepackage{natbib}
\lstset{
keywordstyle=\color{blue},
commentstyle=\color{red},stringstyle=\color[rgb]{0,.5,0},
literate={~}{$\sim$}{1},
basicstyle=\ttfamily\small,
columns=fullflexible,
breaklines=true,
breakatwhitespace=false,
numbers=left,
numberstyle=\ttfamily\tiny\color{gray},
stepnumber=1,
numbersep=10pt,
backgroundcolor=\color{white},
tabsize=4,
keepspaces=true,
showspaces=false,
showstringspaces=false,
xleftmargin=.23in,
frame=single,
basewidth={0.5em,0.4em},
}
\usepackage{natbib, dsfont, pgfpages, tikz,amssymb, amsmath,xcolor}
\bibliographystyle{abbrvnat}
% New operators and commands
\newcommand{\Z}{\mathbb{Z}}
\newcommand{\Q}{\mathbb{Q}}
\newcommand{\R}{\mathbb{R}}
\newcommand{\N}{\mathbb{N}}
\newcommand{\C}{\mathbb{C}}
\renewcommand{\S}{\mathbb{S}}
\newcommand{\blank}{\makebox[1ex]{\textbf{$\cdot$}}}
\newcommand\independent{\protect\mathpalette{\protect\independenT}{\perp}}
\def\independenT#1#2{\mathrel{\rlap{$#1#2$}\mkern2mu{#1#2}}}
\renewcommand{\phi}{\varphi}
\renewcommand{\epsilon}{\varepsilon}
\newcommand*\diff{\mathop{}\!\mathrm{d}}
\newcommand{\weakly}{\rightsquigarrow}
\newcommand\smallO{
  \mathchoice
    {{\scriptstyle\mathcal{O}}}% \displaystyle
    {{\scriptstyle\mathcal{O}}}% \textstyle
    {{\scriptscriptstyle\mathcal{O}}}% \scriptstyle
    {\scalebox{.6}{$\scriptscriptstyle\mathcal{O}$}}%\scriptscriptstyle
}
\newcommand{\midd}{\; \middle|\;}
\newcommand{\1}{\mathds{1}}
\usepackage{ifthen} %% Empirical process with default argument
% \newcommand{\G}[1][]{%
%    \ifthenelse{ \equal{#1}{} }
%       {\ensuremath{\mathbb{G}_n}}
%       {\ensuremath{\mathbb{G}_{#1}}}
% }
% New version:
\newcommand{\G}[2][n]{
{\ensuremath{\mathbb{G}_{#1}}{\left[#2\right]}}
}
\DeclareMathOperator*{\argmin}{\arg\!\min}

% New operators for consistent notation
\newcommand{\V}{\mathrm{Var}} % variance
\newcommand{\measure}[1]{\mathrm{{#1}}} % measure
% \newcommand{\measure}[1]{\textnormal{\textbf{{#1}}}} % measure
\newcommand{\m}[1]{\measure{#1}} % measure shortcut
\newcommand{\eqd}{\stackrel{d}{=}} % equality in distribution
\newcommand{\arrow}[1]{\xrightarrow{\; {#1} \;}}
\newcommand{\arrowP}{\xrightarrow{\; \m{P} \;}} % convergence in probability
\newcommand{\leb}{\lambda} % the Lebesgue measure
\newcommand{\T}{\top} % transpose
\newcommand{\KL}{\ensuremath{D_{\mathrm{KL}}}}

\usepackage{xargs}
% Make it easy to change counterfactual notation:
\newcommandx{\cf}[4][3={}, 4={}]{
  % \ifthenelse{ \equal{#4}{} }
  % {{#1^{#2}}(#3)}
  {\ifthenelse{ \equal{#3}{} }
    {{#1^{#2}}_{#4}}
    {{#1^{#2}}_{#4}(#3)}}
}

% Easily change notation:
\DeclareMathOperator{\TT}{\Psi} % target parameter
\newcommand{\lp}{\mathcal{L}_{\P}^2} % shortcut for lp2 space
\newcommand{\empmeas}{\hat{\mathbb{P}}_n} % empirical measure
\DeclareMathOperator{\E}{\mathbb{E}} % expectation
\renewcommand{\P}{\m{P}} % probability
\newcommand{\ic}{\mathrm{IF}} % influence curve
\setbeamertemplate{footline}[frame number]
\beamertemplatenavigationsymbolsempty
\usepackage{appendixnumberbeamer}
\setbeamercolor{gray}{bg=white!90!black}
\setbeamertemplate{itemize items}{$\circ$}

\renewcommand*\familydefault{\sfdefault}
\itemsep2pt
\usepackage[utf8]{inputenc}
\usepackage[T1]{fontenc}
\usepackage{graphicx}
\usepackage{longtable}
\usepackage{wrapfig}
\usepackage{rotating}
\usepackage[normalem]{ulem}
\usepackage{amsmath}
\usepackage{amssymb}
\usepackage{capt-of}
\usepackage{hyperref}
\usetheme{default}
\author{Anders Munch \newline \small joint work with Thomas G., Helene, and Mark van der Laan}
\date{\today}
\title{Some comments about the Highly-Adaptive LASSO with focus on survival data}
\begin{document}

\maketitle
\begin{frame}{Outline}
\tableofcontents
\end{frame}

\section{Setting and motivation}
\label{sec:org4797d1c}
\begin{frame}[label={sec:org92f6536}]{Motivation}
\small
\begin{block}{Reasons for using HAL}
\begin{itemize}
\item Theoretically important
\item ``Dimension-free'' convergence rate
\item Few assumptions imposed
\end{itemize}

\pause  
\end{block}

\begin{block}{Better understanding of the estimator}
\begin{itemize}
\item Multivariate càdlàg functions
\item Sectional variation norm
\end{itemize}

\pause
\end{block}

\begin{block}{Theoretical discussion}
\begin{itemize}
\item In practice we will have to approximate the HAL estimator for computational
reasons
\item Still nice to know that the estimator we approximate has  nice properties
\end{itemize}
\end{block}
\end{frame}


\section{Functions of bounded sectional variation norm}
\label{sec:orga7c3652}
\begin{frame}[label={sec:org12fba9a}]{Definition of the estimator}
\small

\begin{description}
\item[{\(\mathcal{D}_M([0,1]^d)\)}] the space of \color{blue}càdlàg \color{black}
functions \(f \colon [0,1]^d \rightarrow \R\) with \color{blue}sectional
variation norm \color{black} bounded by \(M\).
\item[{\(\mathcal{O}\)}] the sample space
\item[{\(L\)}] a loss function, \(L(f, O) \in \R_+\)
\end{description}

\vfill 

The parameter of interest is the function minimizing the expected loss (risk)
\begin{equation*}
  f_0 = \argmin_{f\in \mathcal{D}_M([0,1]^d)} P[L(f, \blank)]
  = \argmin_{f\in \mathcal{D}_M([0,1]^d)} \int_{\mathcal{O}} L(f, o)  P (\diff o).
\end{equation*}

\vfill 

We estimate \(f_0\) with the function minimizing the empirical risk
\begin{equation*}
  \hat{f}_n
  = \argmin_{f\in \mathcal{D}_M([0,1]^d)} \empmeas[L(f, \blank)]
  = \argmin_{f\in \mathcal{D}_M([0,1]^d)} \frac{1}{n}\sum_{i=1}^{n}L(f,O_i).
\end{equation*}
\end{frame}

\begin{frame}[label={sec:orgddd7a6b}]{Multivariate càdlàg functions}
\small Which direction is ``left'' when \(d>1\)? \pause \vfill

\begin{definition}[\cite{neuhaus1971weak}]
For a point \(u \in [0,1]^d\) and a vertex $\textbf{a} \in \{0,1\}^d$ look at
quadrants \(Q_{\textbf{a}}(u)\) spanned by \(u\) and \(\textbf{a}\). The limit
of \(f(u_n)\) for \(\{u_n\} \subset Q_{\textbf{a}}(u)\), \(u_n \rightarrow u\)
should exist, and if \(\textbf{a} = (1, 1, \dots, 1)\) then
\(\lim_{n\rightarrow\infty}f(u_n) = f(u)\).

\hfill
\end{definition}

\begin{onlyenv}<1-2>
\begin{center}
\includegraphics[width=.45\textwidth]{/tmp/babel-rOvgph/figure-C3B1q9.pdf}
\end{center}
\end{onlyenv}

\begin{onlyenv}<3>
\begin{center}
\includegraphics[width=.45\textwidth]{/tmp/babel-rOvgph/figure-IWZuos.pdf}
\end{center}
\end{onlyenv}

\begin{onlyenv}<4>
\begin{center}
\includegraphics[width=.45\textwidth]{/tmp/babel-rOvgph/figure-DoBTJf.pdf}
\end{center}
\end{onlyenv}

\begin{onlyenv}<5>
\begin{center}
\includegraphics[width=.45\textwidth]{/tmp/babel-rOvgph/figure-PJlUg9.pdf}
\end{center}
\end{onlyenv}


\begin{onlyenv}<6>
\begin{center}
\includegraphics[width=.45\textwidth]{/tmp/babel-rOvgph/figure-yqrcdJ.pdf}
\end{center}
\end{onlyenv}
\end{frame}


\begin{frame}[label={sec:orgd741d85}]{Càdlàg function with jumps}
If we picture ``continuous from the right with left-hand limits'' in dimension
\(d=1\) this does not seem very restrictive -- for instance, ``jumps'' are allowed.\pause

\begin{alertblock}{\center Don't trust \(d=1\)!}
\end{alertblock}

\begin{onlyenv}<1-2>
\begin{center}
\includegraphics[width=.9\linewidth]{/tmp/babel-rOvgph/figure-rC1Ra9.pdf}
\end{center}
\end{onlyenv}


\begin{onlyenv}<3>
\begin{center}
\includegraphics[width=.9\linewidth]{/tmp/babel-rOvgph/figure-jx8plX.pdf}
\end{center}
\end{onlyenv}

\begin{onlyenv}<4>
\begin{center}
\includegraphics[width=.9\linewidth]{/tmp/babel-rOvgph/figure-l9e5sH.pdf}
\end{center}
\end{onlyenv}

\begin{onlyenv}<5>
\begin{center}
\includegraphics[width=.9\linewidth]{/tmp/babel-rOvgph/figure-85pO7q.pdf}
\end{center}
\end{onlyenv}
\end{frame}

\begin{frame}[label={sec:org45c949e}]{Sectional variation norm}
\pause
In dimension $d=1$, the variation norm of a function is
\begin{equation*}
  \Vert f \Vert_{v} = \sup_{\pi} \sum_{i=1}^{|\pi|} |f(t_{i})-f(t_{i-1})|,
\end{equation*}
where the supremum is taken over all finite partitions
\(0=t_0 < t_1 < \dots < t_{\pi} = 1\).

\vfill \pause

At first sight, a natural generalization seems to be the \emph{Vitali variation}:
\begin{equation*}
  V^{(d)}(f) = \sup_{\pi} \sum_{A \in \pi} | \Delta(f;A)|,
\end{equation*}
where the supremum is taken over all ``grid partitions'' and \(\Delta(f;A)\) is
the \textit{quasi-volume} that \(f\) assigns the rectangle $A$.
\end{frame}


\begin{frame}[label={sec:org2588226}]{Vitali variation: \normalsize \(V^{(d)}(f) = \sup_{\pi} \sum_{A \in \pi} | \Delta(f;A)|\)}
\begin{onlyenv}<2>
\small
\color{white}
\begin{equation*}
  \text{When \(d=2\),} \quad 
  \Delta(f;A) = f(b_1, b_2) - f(b_1, a_2) - f(a_1, b_2) + f(a_1, a_2).
\end{equation*}
\color{black}

\vfill

\begin{center}
\includegraphics[width=.9\linewidth]{/tmp/babel-rOvgph/figure-Sgklgx.pdf}
\end{center}
\end{onlyenv}

\begin{onlyenv}<3>
\small
\color{black}
\begin{equation*}
  \text{When \(d=2\),} \quad 
  \Delta(f;A) = f(b_1, b_2) - f(b_1, a_2) - f(a_1, b_2) + f(a_1, a_2).
\end{equation*}
\color{black}

\vfill

\begin{center}
\includegraphics[width=.9\linewidth]{/tmp/babel-rOvgph/figure-YYGyFB.pdf}
\end{center}
\end{onlyenv}

\begin{onlyenv}<4>
\small
\color{black}
\begin{equation*}
  \text{When \(d=2\),} \quad 
  \Delta(f;A) = f(b_1, b_2) - f(b_1, a_2) - f(a_1, b_2) + f(a_1, a_2).
\end{equation*}
\color{black}

\vfill

\begin{center}
\includegraphics[width=.9\linewidth]{/tmp/babel-rOvgph/figure-OWE0WT.pdf}
\end{center}
\end{onlyenv}

\begin{onlyenv}<5>
\small
\color{black}
\begin{equation*}
  \text{When \(d=2\),} \quad 
  \Delta(f;A) = f(b_1, b_2) - f(b_1, a_2) - f(a_1, b_2) + f(a_1, a_2).
\end{equation*}
\color{black}

\vfill

\begin{center}
\includegraphics[width=.9\linewidth]{/tmp/babel-rOvgph/figure-KGylG2.pdf}
\end{center}
\end{onlyenv}

\begin{onlyenv}<6>
\small
\color{black}
\begin{equation*}
  \text{When \(d=2\),} \quad 
  \Delta(f;A) = f(b_1, b_2) - f(b_1, a_2) - f(a_1, b_2) + f(a_1, a_2).
\end{equation*}
\color{black}

\vfill

\begin{center}
\includegraphics[width=.9\linewidth]{/tmp/babel-rOvgph/figure-aYiSCu.pdf}
\end{center}
\end{onlyenv}
\end{frame}


\begin{frame}[label={sec:org2892906}]{Sectional variation norm (Hardy-Krause variation)}
However, many interesting functions have Vitalli variation equal to 0, e.g., the
function $f(x,y) = x$.

\vfill \pause We should also look at the function on the \textit{faces}
\begin{equation*}
  U_s = \{(u_1, \dots, u_d) \in [0,1]^d : u_j = 0, j \not\in s \},
\end{equation*}
for non-empty subsets \(s \subset \{1, \dots, d\}\).

\vfill \pause
We denote by \(f_s \colon [0,1]^{|s|} \rightarrow \R\) the restriction of \(f\)
to \(U_s\) and define the norm
\begin{equation*}
  \Vert f \Vert_{v} = \sum_{s} V^{(|s|)}(f_s),
\end{equation*}
where the sum is taken over all non-empty subsets \(s \subset \{1, \dots, d\}\).

\vfill

This is referred to as the \emph{Hardy-Krause variation} by
\cite{fang2021multivariate} and the \emph{sectional variation norm} by
\cite{van2017generally}.
\end{frame}


\begin{frame}[label={sec:org3aa0e08}]{The sectional variation norm of smooth functions}
In $d=1$, if $f$ is differentiable then
\begin{equation*}
  \Vert f \Vert_{v} = \int_0^1 |f'(x)| \diff x.
\end{equation*}
\pause $\rightarrow$ Mild regularity condition. 

\pause

\begin{alertblock}{\center Don't trust \(d=1\)!}
\hfill\pause

In $d>1$, if $f$ is sufficiently smooth then
\begin{equation*}
  \Vert f \Vert_{v} = \sum_{s} \int_0^1 \cdots \int_0^1
  \left\vert
    \frac{\partial^{|s|} f }{\partial x_1 \cdots \partial x_{|s|}}
  \right\vert\diff x_1 \dots x_{|s|}.
\end{equation*}

\hspace{0.5cm}

\pause $\rightarrow$ Constraints on all mixed derivatives of order less than or
equal to \(d\).

\hspace{0.5cm}

\pause $\rightarrow$ The sum contains \(\sum_{k=1}^{n} {n \choose k} = (2^d-1)\)
terms.
\end{alertblock}
\end{frame}



\begin{frame}[label={sec:org2c09576}]{Implementation of the estimator}
\pause \small \cite{gill1995inefficient} and \cite{van2017generally} give the
following representation of any \(f\in \mathcal{D}_M([0,1]^d)\):
\begin{equation*}
    f(x) =
          \int_{[0, x]} \diff f
 = f(0)
          + \sum_{s} \int_{(0_s, x_s]} \diff f_s.
\end{equation*}

\vfill
\pause

The norm $\Vert f \Vert_{v}$ is equal to the sum of the total variation of the
measures on $(0(s), 1(s)]$ generated by the section $f_s$ of $f$ -- hence the
name.

\vfill
\pause

Suggests estimating \(f\) by estimating the measures \(\diff f_s\) with weighted
empirical measures in the sections \(s\):
\begin{equation*}
  f_{\beta} = \beta_0 +
  \sum_{s}\sum_{i=1}^{n}\beta_{i,s} \psi_{i,s}(x),
  \quad \text{with} \quad
  \psi_{i,s}(x)=
  \1\{X_i(s) \leq x(s)\},
\end{equation*}
with \(\Vert\beta\Vert_1 = \sum |\beta_{s,i}| \leq M\).

\vfill
\pause

This can be phrased as the LASSO problem
\begin{equation*}
 \argmin_{\beta }
  \empmeas{[L(f_{\beta}, \blank)]},
  \quad \text{such that} \quad
  \Vert \beta \Vert_1 \leq M.
\end{equation*}
\end{frame}

\begin{frame}[label={sec:org199b560}]{Basis functions for the HAL estimator}
\begin{onlyenv}<1>
\begin{center}
\includegraphics[width=.9\linewidth]{/tmp/babel-rOvgph/figure-wJjow6.pdf}
\end{center}
\end{onlyenv}

\begin{onlyenv}<2>
\begin{center}
\includegraphics[width=.9\linewidth]{/tmp/babel-rOvgph/figure-H2rg2d.pdf}
\end{center}
\end{onlyenv}

\begin{onlyenv}<3>
\begin{center}
\includegraphics[width=.9\linewidth]{/tmp/babel-rOvgph/figure-Mcrndj.pdf}
\end{center}
\end{onlyenv}

\begin{onlyenv}<4>
\begin{center}
\includegraphics[width=.9\linewidth]{/tmp/babel-rOvgph/figure-vkJ45q.pdf}
\end{center}
\end{onlyenv}

\begin{onlyenv}<5>
\begin{center}
\includegraphics[width=.9\linewidth]{/tmp/babel-rOvgph/figure-Rywnq6.pdf}
\end{center}
\end{onlyenv}

\begin{onlyenv}<6>
\begin{center}
\includegraphics[width=.9\linewidth]{/tmp/babel-rOvgph/figure-O2X9Lf.pdf}
\end{center}
\end{onlyenv}

\begin{onlyenv}<7>
\begin{center}
\includegraphics[width=.9\linewidth]{/tmp/babel-rOvgph/figure-dj0Wj4.pdf}
\end{center}
\end{onlyenv}

\begin{onlyenv}<8>
\begin{center}
\includegraphics[width=.9\linewidth]{/tmp/babel-rOvgph/figure-5jmbHN.pdf}
\end{center}
\end{onlyenv}
\end{frame}


\section{Some challenges with the exact definition of the estimator}
\label{sec:org538056c}
\begin{frame}[label={sec:org66f28f3}]{The solution to the minimization problem}
\small \vspace{-.5cm}
\begin{equation*}
 \hat{\beta}_n=\argmin_{\beta : \Vert \beta \Vert_1 \leq M}
  \empmeas{[L(f_{\beta}, \blank)]},
  \quad \text{with} \quad 
  f_{\beta} =\beta_0 +
  \sum_{s}\sum_{i=1}^{n}\beta_{i,s} \psi_{i,s}(x).
\end{equation*}
\pause It seems to be common wisdom that
\begin{equation*}
  f_{\hat{\beta}_n} = \hat{f}_n= \argmin_{f \in \mathcal{D}_M{([0,1]^d)}} \empmeas{[L(f, \blank)]}.
\end{equation*}

\pause \vfill

\begin{block}{\normalsize \(\checkmark\) when \(d=1\) and, for instance, \(L(f,(X,Y)) =  \{f(X)-Y\}^2\)}
\small \pause

Given \(f\), construct \(\bar{f}\) as the piece-wise constant function such that
\(\bar{f}(X_i) = f(X_i)\) and \(\bar{f}(0) = f(0)\). \pause Then
\begin{equation*}
  L(f, (X_i, Y_i)) =L(\bar{f}, (X_i, Y_i)),
  \quad \text{for all} \quad i =1, \dots,n,
\end{equation*}
\pause and, with \(0=X_{(0)} \leq X_{(1)} \cdots \leq X_{(n)}\),
\begin{align*}
  \Vert \bar{f} \Vert_{v}
  =
  \sum_{i=1}^{n} |\bar{f}(X_{(i)})-\bar{f}(X_{(i-1)})|
  & =
    \sum_{i=1}^{n} |f(X_{(i)})-f(X_{(i-1)})|
  \\
  &
    \leq
    \sup_{\pi} \sum_{i=1}^{|\pi|} |f(t_{i})-f(t_{i-1})|
    = \Vert f \Vert_{v}.
\end{align*}
\end{block}
\end{frame}

\begin{frame}[label={sec:org83bb0e9}]{Not clear same trick works when \(d>1\) \pause -- \color{red} don't trust \(d=1\)!}
\begin{onlyenv}<3>
\phantom{We have $6\times 3 +1 = 17$ basis functions but $(6+1)^2 = 49$ rectangles.}
\begin{center}
\includegraphics[width=.9\linewidth]{/tmp/babel-rOvgph/figure-MmKJKt.pdf}
\end{center}
\end{onlyenv}

\begin{onlyenv}<4>
We have \(6\times 3 +1 = 17\) basis functions but \((6+1)^2 = 49\) rectangles.
\begin{center}
\includegraphics[width=.9\linewidth]{/tmp/babel-rOvgph/figure-JNtasN.pdf}
\end{center}
\end{onlyenv}


\begin{onlyenv}<5->
\cite{fang2021multivariate} formally show that when \(L\) is the squared error
loss, the minimizer
\begin{equation*}
  \hat{f}_n= \argmin_{f \in \mathcal{D}_M{([0,1]^d)}} \empmeas{[L(f, \blank)]}
\end{equation*}
can be taken to be the solution to a LASSO problem using indicator functions as
basis functions.

\vfill

\begin{beamercolorbox}[rounded=true]{gray}
However, they need up to $\asymp {\color{red}n^d}$ basis functions whereas the HAL estimator is made
up of only $n \times (2^d-1) + 1 \asymp {\color{red}n}$ basis functions.
\end{beamercolorbox}
\end{onlyenv}
\end{frame}


\begin{frame}[label={sec:orgc7aa811}]{The estimator \(\hat{f}_n\) in the survival setting}
\pause

Consider estimation of the hazard for the survival time \(T\).

\vfill

\begin{description}
\item[{Data}] \(O=(\tilde{T}, \Delta)\), \(\tilde{T} = T \wedge C\), \(\Delta \in \{0,1\}\)
\item[{Hazard}] \(h = e^{f}\), \(f \in \mathcal{D}_M{([0,1])}\)
\item[{Loss}] \(L(f, O) = \int_0^{\tilde{T}} e^{f(s)} - \Delta f(\tilde{T})\)
\end{description}

\vfill
\pause

\begin{beamercolorbox}[rounded=true]{gray}
If there is an \(i \in \{1, \dots, n-1\}\) such that \(f(\tilde{T}_{(i)}) >
f(\tilde{T}_{(i+1)})\) then \(f\) is not the minimizer of \(\empmeas{[L(f,\blank)]}\)
over \(\mathcal{D}_M{([0,1])}\).

\pause
\end{beamercolorbox}

\begin{block}{\color{white} .}
\begin{description}
\item[{\(\implies\)}] The empirical risk minimizer \(\hat{f}_n\) is in general either not
well-defined or a very bad estimator.
\end{description}
\end{block}
\end{frame}


\begin{frame}[label={sec:org74360b4}]{Proof by picture}
\begin{equation*}
  L(f, O_i) = \int_0^{\tilde{T}_i} e^{f(s)} - \Delta_i f(\tilde{T}_i),
  \qquad
  \empmeas{[L(f, \blank)]} = \frac{1}{n}\sum_{i=1}^{n} L(f, O_i)
\end{equation*}

\vfill

\begin{onlyenv}<1>
\begin{center}
\includegraphics[width=.9\linewidth]{/tmp/babel-rOvgph/figure-rVP0Rt.pdf}
\end{center}
\end{onlyenv}

\begin{onlyenv}<2>
\begin{center}
\includegraphics[width=.9\linewidth]{/tmp/babel-rOvgph/figure-uhLFpX.pdf}
\end{center}
\end{onlyenv}

\begin{onlyenv}<3>
\begin{center}
\includegraphics[width=.9\linewidth]{/tmp/babel-rOvgph/figure-ghI2ny.pdf}
\end{center}
\end{onlyenv}
\end{frame}


\section{Approximate minimization is sufficient}
\label{sec:org8577335}
\begin{frame}[label={sec:orgfdcb0ce}]{\color{white} break title}
\begin{block}{\center \Large Need results for \(f_{\hat{\beta}_n}\) instead of \(\hat{f}_n\)}
\vspace{2cm}
\pause
(Note that \(f_{\hat{\beta}_n}\) is well-defined in the survival setting.)
\end{block}
\end{frame}

\begin{frame}[label={sec:org4d131b4}]{Deriving convergence rates using empirical processes theory}
\pause \small
The convergence rate for an empirical loss minimizer over a function space
$\mathcal{F}$ can be read off from the \textit{modulus of continuity} of the
empirical process \(\mathbb{G}_n = \sqrt{n}(\empmeas-P)\) over the space
\begin{equation*}
  \mathcal{L} = \{L(f, \blank) - L(f_0, \blank) : f \in \mathcal{F}\},
\end{equation*}
which is defined as
\begin{equation*}
  \phi_n(\delta) = \E\left[ \Vert \mathbb{G}_n \Vert_{\mathcal{L}(\delta)} \right],
  \quad \text{where} \quad
  \Vert \mathbb{G}_n \Vert_{\mathcal{L}(\delta)}
  = \sup_{h \in {\mathcal{L}(\delta)}} |\mathbb{G}_n{[h]}|,
\end{equation*}
and \(\mathcal{L}(\delta) = \{h \in \mathcal{L}: \Vert h \Vert \leq \delta \}\).

\vfill \pause

The modulus $\phi_n$ can be controlled by the covering or bracketing entropy for
$\mathcal{F}$. When $\mathcal{F}= \mathcal{D}_M{([0,1]^d)}$ this leads to the
convergence rate
\begin{equation*}
  \Vert \hat{f}_n - f_0 \Vert = \smallO_P(r_n),
  \quad \text{for} \quad 
  r_n = n^{-1/3}\log(n)^{2(d-1)/3}.
\end{equation*}

\vfill \pause

{\color{blue}\textbf{Exact minimization is not needed}} -- we just need the
estimator \(f_n^{\star}\) to fulfill
\begin{equation*}
  \empmeas{[L(f^{\star}_n, \blank)]} \leq \empmeas{[L(f_0, \blank)]}  + \mathcal{O}_P{(r_n^2)}.
\end{equation*}
\end{frame}

\begin{frame}[label={sec:org2d326ff}]{This holds for \large \(f_{\hat{\beta}_n}\): \(\empmeas{[L(f_{\hat{\beta}_n}, \blank)]} \leq \empmeas{[L(f_0, \blank)]}  + \mathcal{O}_P{(r_n^2)}\)}
\small\pause
Write
\begin{equation*}
  f_0(x) = f_0(0)
  + \sum_{s} \int_{(0_s, x_s]} \diff f_{0,s}
  = f_0(0)
  + \sum_{s} \int_{(0_s, x_s]} \frac{\diff f_{0,s}}{\diff P_s} \diff P_s,
\end{equation*}
and define
\begin{equation*}
  \tilde{f}_n = f_0(0)
  + \sum_{s} \int_{(0_s, x_s]} \frac{\diff f_{0,s}}{\diff P_s} \diff
  \mathbb{P}_{s,n}.
\end{equation*}

\vfill \pause

The function \(\tilde{f}_n\) is on the form
\( f_{\beta} = \beta_0 + \sum_{s}\sum_{i=1}^{n}\beta_{i,s} \psi_{i,s}(x)\), and
by the law of large numbers
\( \Vert \tilde{f}_n \Vert_{v} \arrow{P} \Vert f_0 \Vert_{v}\).

\vfill\pause Hence if \(\Vert f_0 \Vert_{v}<M\) then
\(\empmeas{[L(f_{\hat{\beta}_n}, \blank)]} \leq \empmeas{[L(\tilde{f}_{n},
  \blank)]}\) with prob. $\rightarrow$ 1, so
\begin{equation*}
  \empmeas{[L(f_{\hat{\beta}_n}, \blank)]} - \empmeas{[L(f_0, \blank)]}
  \leq
  \empmeas{[L(\tilde{f}_{n},
  \blank)]}
- \empmeas{[L(f_0, \blank)]}
\quad \text{with prob.\ } \rightarrow 1.
\end{equation*}

\vfill \pause

\begin{equation*}
  \Vert \tilde{f}_n - f_0 \Vert_{\infty} = 
  n^{-1/2}
  \sum_{s} \Vert \mathbb{G}_{s,n}
  \Vert_{\mathcal{D}}
  = \mathcal{O}_p{(n^{-1/2})}.  
\end{equation*}

\vfill \pause Combine this with a bound on \(\phi_n(\delta)\) for
\(\mathcal{D}_M{([0,1]^d)}\) to obtain
\begin{equation*}
  \empmeas{[L(\tilde{f}_{n}, \blank)]}
  - \empmeas{[L(f_0, \blank)]} =
  \mathcal{O}_P{(r_n^2)}.
\end{equation*}
\end{frame}

\begin{frame}[label={sec:org1f4f1cf}]{Conclusion and discussion}
\begin{itemize}[<+->]
\item We obtain the wanted convergence rate when using \(f_{\hat{\beta}_n}\) as our
estimator instead of \(\hat{f}_n\).
\item This is an \emph{asymptotic} result -- no focus on finite sample bounds.
\item Would \(\hat{f}_n\) (when it is defined) perform better than
\(f_{\hat{\beta}_n}\)? -- or is the reduction in the number of basis functions
actually attractive in finite samples?
\item Can we reduce the number of basis functions further? -- would be
computationally attractive.
\item What kind of constraints are put on functions in
\(\mathcal{D}_M{([0,1]^d)}\) that are continuous but not much smoother?
\end{itemize}
\end{frame}

\section*{References}
\label{sec:orgd231cbc}
\begin{frame}[label={sec:orgbc2cc3c}]{References}
\footnotesize \bibliography{./latex-settings/default-bib.bib}
\end{frame}
\end{document}